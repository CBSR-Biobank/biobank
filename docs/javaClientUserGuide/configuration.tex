\chapter{Configuration}
The configuration settings for the Java Client can be found by selecting
\texttt{Configuration -> Preferences} from the main menu.
    \begin{figure}[H]
      \centering
      \scalebox{0.5}
      { \includegraphics*{screenshots/configuration/preferences_dialog} }
      \caption{Configuration preferences dialog box.}
      \label{fig:preferences_dialog}
    \end{figure}
A configuration page can be selected by clicking on a node on the
tree on the left hand side of the dialog box. The \texttt{filter text} box can
be used to quickly find a configuration page by typing the name of the page in
whole or in part.

The following sections discuss the configuration pages in detail.

\section{Automatic Updates}
The Java Client has the capability of detecting when a new version of the
software has been made available by the software design team. The settings on
this page allow the user to specify when updates will be searched for, when to
download them, and how to notify the user when they are found.

\section{General}
The settings on this page are general to the application.
\begin{itemize}
  \item The Java Client's software version can be displayed in the window's
    title bar by checking this box.\textit{Note that this option is only
      required by technicians when they are testing new versions of the
      software}.
  \item The \emph{heap status} can be displayed in the main window by checking
    this box. The heap status gives an indication of the memory used by the
    application. \textit{This option is in place for the sofware designers and
      is not required for normal use}.
\end{itemize}

\section{Issue Tracker}
The application has the capability of sending \emph{Problem Report} emails to
the software design team\footnote{\texttt{Help -> Send Error Email} from the
  main menu}. The error emails have application state information attached to
aid team in resolving potential problems. The settings on this page are the
settings used to send the emails to the team.

\textit{User should not modify these settings unless requested to by the software
  design team}.
\begin{center}
\begin{tabular}{|l|p{3.5in}|}
  \hline
  \textbf{Tracker email} & The email address to send problem reports to.\\
  \hline
  \textbf{SMTP server} & The SMTP mail server to use when sending emails.\\
  \hline
  \textbf{SMPT server port} & The SMTP server port.\\
  \hline
  \textbf{SMTP server username} & The user name to use with the SMTP server.\\
  \hline
  \textbf{SMPT server passowrd} & The password to use with the SMTP server.\\
  \hline
\end{tabular}
\end{center}


\section{Link \ Assign}
These settings are used when scanning and decoding pallets and linking them to
patients or container locations.

To speed up processing of samples a handheld barcode scanner can be used to
quickly enter information into the application.

\section{Scanning and Decoding}
The settings on this page allow the user to specify the scanning and
decoding parameters used when decoding pallet images.
\begin{center}
\begin{tabular}{|l|p{5in}|}
  \hline
  \textbf{Select Scanner} & Used to select the scanner that will be used to scan
    96 well pallets.\\
  \hline
  \textbf{Driver Type} & The type of driver that was selected when the \emph{Select
    Scanner} button was pressed. Normally, the application will attempt to
    determine the driver type as soon as the user makes the selection, but
    sometimes the application does not select the correct type. Use the
    checkboxes here to override what the application selected.\\
  \hline
  \textbf{DPI} & The \emph{Dots per Inch} to scan images at. For best results use
    600 DPI.\\
  \hline
  \textbf{Brightness} & The brightness setting to be used when scanning
    images. This parameter does not work on Hewlett-Packard scanners when using
    the WIA based driver.\\
  \hline
  \textbf{Contrast} & The contrast setting to be used when scanning
    images. This parameter does not work on Hewlett-Packard scanners when using
    the WIA based driver.\\
  \hline
\end{tabular}
\end{center}
\subsection{Decoding Parameters}

\begin{center}
\begin{tabular}{|l|p{4in}|}
  \hline
  \textbf{Decode Library Debug Level} & The decoding software library can
  output debugging information to a log file. When this value is zero there is
  no debugging information stored in the log file. Possible values are 0
  through 9. The higher the value the more detailed the debugging information.\\
  \hline

  \textbf{Decode Edge Threshold} & Set the minimum edge threshold as a
  percentage of maximum. For example, an edge between a pure white and pure
  black pixel would have an intensity of 100.  Edges with intensities below the
  indicated threshold will be ignored by the decoding process. Lowering the
  threshold will increase the amount of work to be done, but may be necessary
  for low contrast or blurry images. The default and recommended value is
  \emph{5}.\\

  \hline

  \textbf{Decode Square Deviation} & Maximum deviation (in degrees) from
  squareness between adjacent barcode sides. The default and recommended value
  is \emph{N=15} and is meant for scanned images. Barcode regions found with
  corners \emph{<(90-N)} or \emph{>(90+N)} will be ignored by the decoder.\\

  \hline
  \textbf{Decode Corrections} &  The number of corrections to make while
  decoding. The defaut and recommended value is 10.\\
  \hline
  \textbf{Decode Scan Gap} & The scan grid gap size in inches. The
  default and recommended value is \emph{0.085}.\\
  \hline
  \textbf{Decode Cell Distance} & The distance in inches between tubes.The
  default and recommended value is \emph{0.345} for NUNC pallets.\\
  \hline
\end{tabular}
\end{center}

\subsection{Decoding Profiles}
\subsection{Plate Positions}
To define a pallet scanning region do the following:
\begin{enumerate}
  \item On the preferences dialog window, if there is a "plus" symbol next to
    the \emph{Scanning and Decoding} node, press it to expand the sub
    tree.
    \begin{figure}[H]
      \centering
      \scalebox{0.5}
      { \includegraphics*{screenshots/configuration/plate1_definition} }
      \caption{Configuring a plate position.}
      \label{fig:plate1_definition}
    \end{figure}
  \item Place a pallet that contains tubes on the flatbed scanner. Ensure the
    top edge of the pallet is touching the top of the scanning region, and the right
    edge of the pallet is touching the right margin. Ensure the 12 columns
    are vertical and the 8 rows are horizontal.
  \item Select the plate region you are going to define.  If it is the first
    select \emph{Plate 1 Position}.
  \item Click on the "Enable" box.
  \item Press the "Scan" button. Now wait for the scanner to scan the entire
    flatbed.
    \begin{figure}[H]
      \centering
      \scalebox{0.5}
      { \includegraphics*{screenshots/configuration/sample_flatbed_scan} }
      \caption{Sample flatbed scan.}
      \label{fig:sample_flatbed_scan}
    \end{figure}
  \item Once the scan is done, You will see something similar to Figure
    \ref{fig:sample_flatbed_scan}. The image shown on the right hand side is
    the image taken by the scanner and superimposed is a grid with 8 rows and
    12 columns. The cell colored in cyan should correspond to tube in row A and
    column 1.
  \item Under orientation select "Landscape".
  \item You can adjust the size of the grid using the mouse. If you move the
    mouse to one of the corners or one of the edges you can resize the grid by
    holding down the left button on the mouse. The whole grid can be moved by
    pressing the left mouse button while hovering inside the grid.
  \item Once the grid cells are aligned with each tube press the "OK" button
    (see Figure \ref{fig:plate1_grid_ready}). The wheel on the mouse can be
    used to make the cells smaller or bigger.
    \begin{figure}[H]
      \centering
      \scalebox{0.5}
      { \includegraphics*{screenshots/configuration/plate1_grid_ready} }
      \caption{Grid aligned with tubes.}
      \label{fig:plate1_grid_ready}
    \end{figure}
  \item Repeat from step 2 to define any more pallet scanning regions.
  \item Usually only one pallet scanning region is required for normal
    operation of the software.
\end{enumerate}
Figure \ref{fig:plate2_grid_ready} shows an example of how \emph{Plate 2} can
be defined. Here Plate 2 is touching the top and the left margin of
the of the flatbed region. The plate on the left of the image is where Plate 1
is defined.
    \begin{figure}[H]
      \centering
      \scalebox{0.5}
      { \includegraphics*{screenshots/configuration/plate2_grid_ready} }
      \caption{Plate 2 grid aligned with tubes.}
      \label{fig:plate2_grid_ready}
    \end{figure}
Note that cell A1 should be at the \emph{Top Left} when configuring a plate in
\textbf{Landscape} orientation and \emph{Top Right} when in \textbf{Portrait}
orientation when looking down at the scanner's flatbed.

To test if your configuration will yield valid decodes use the
\texttt{Scanner -> Decode Plate} from the main menu.
\subsection{Plate Barcodes}
\section{Servers}
