\chapter{Configuration}
\section{General}
\section{Issue Tracker}
\section{Link \ Assign}
\section{Scanning and Decoding}
The items on this preferences page allow the user to specify the scanning and
decoding parameters used to decode 2D barcodes on aliquot tubes.
\begin{center}
\begin{tabular}{|l|p{5in}|}
  \hline
  \textbf{Select Scanner} & Used to select the scanner that will be used to scan
    96 well pallets.\\
  \hline
  \textbf{Driver Type} & The type of driver that was selected when the \emph{Select
    Scanner} button was pressed. Normally, the application will attempt to
    determine the driver type as soon as the user makes the selection, but
    sometimes the application does not select the correct type. Use the
    checkboxes here to override what the application selected.\\
  \hline
  \textbf{DPI} & The \emph{Dots per Inch} to scan images at. For best results use
    600 DPI.\\
  \hline
  \textbf{Brightness} & The brightness setting to be used when scanning
    images. This parameter does not work on Hewlett-Packard scanners when using
    the WIA based driver.\\
  \hline
  \textbf{Contrast} & The contrast setting to be used when scanning
    images. This parameter does not work on Hewlett-Packard scanners when using
    the WIA based driver.\\
  \hline
\end{tabular}
\end{center}
\subsection{Decoding Parameters}

\begin{center}
\begin{tabular}{|l|p{4in}|}
  \hline
  \textbf{Decode Library Debug Level} & The decoding software library can
  output debugging information to a log file. When this value is zero there is
  no debugging information stored in the log file. Possible values are 0
  through 9. The higher the value the more detailed the debugging information.\\
  \hline

  \textbf{Decode Edge Threshold} & Set the minimum edge threshold as a
  percentage of maximum. For example, an edge between a pure white and pure
  black pixel would have an intensity of 100.  Edges with intensities below the
  indicated threshold will be ignored by the decoding process. Lowering the
  threshold will increase the amount of work to be done, but may be necessary
  for low contrast or blurry images. The default and recommended value is
  \emph{5}.\\

  \hline

  \textbf{Decode Square Deviation} & Maximum deviation (in degrees) from
  squareness between adjacent barcode sides. The default and recommended value
  is \emph{N=15} and is meant for scanned images. Barcode regions found with
  corners \emph{<(90-N)} or \emph{>(90+N)} will be ignored by the decoder.\\

  \hline
  \textbf{Decode Corrections} &  The number of corrections to make while
  decoding. The defaut and recommended value is 10.\\
  \hline
  \textbf{Decode Scan Gap} & The scan grid gap size in inches. The
  default and recommended value is \emph{0.085}.\\
  \hline
  \textbf{Decode Cell Distance} & The distance in inches between tubes.The
  default and recommended value is \emph{0.345} for NUNC pallets.\\
  \hline
\end{tabular}
\end{center}

\subsection{Decoding Profiles}
\subsection{Plate Positions}
In order to decode the 2D barcodes on the aliquot tubes, the plate scanning
regions must be configured in the preferences. You must first select the
scanning parameters and then define the regions that will contain the pallets
on the scanner.
\subsection{Plate Barcodes}
\section{Servers}
