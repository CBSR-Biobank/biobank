\chapter{Overview}
The BioBank2 Java Client is designed to be a client application that connects
to a server which provides services for storing inventory information for
biological samples. More than one client can connect to the server at the same
time. The client is a software application that runs on Microsoft Windows and
Linux\footnote{The scanner cannot be used when using the application on Linux.}. The client
software is a Java application and requires that a Java JRE or JDK be installed
on the computer. The installation program for the client installs a Java JRE if
it is not already present on the computer.

When running on Microsoft Windows, the client can connect to scanners use either
TWAIN\footnote{\url{http://en.wikipedia.org/wiki/TWAIN}} or
WIA\footnote{\url{http://en.wikipedia.org/wiki/Windows_Image_Acquisition}}
drivers. Most scanners available on the market provide these
drivers. Therefore, most scanners work with the client software.

Figure \ref{fig:main_window} shows the application's main window and highlights
different some of its components.
    \begin{figure}[H]
      \centering
      \scalebox{0.4}
      { \includegraphics*{screenshots/overview/main_window} }
      \caption{The Java Client's main window.}
      \label{fig:main_window}
    \end{figure}
\begin{description}
  \item[Main Menu] The main menu allows the user access to the different
    functions provided by the software (see section \ref{sec:main_menu} for a
    description of the menu items). The application also has an icon based
    toolbar to access commonly used functions.
  \item[Toolbar Icons] The icon buttons in the toolbar allow quick access to
    often used menu items. Using the mouse, the user can hover over the icon
    button and after 2 seconds a tooltip is displayed describing what the
    button is for and if applicable the keyboard shortcut. Note that there is
    an icon toolbar under the Main Menu and another grouped with the tree view.
  \item[Tree View] The tree view shown here is for the \emph{Administration
    View}. The application has multiple views and are described in section
    \ref{sec:application_views}.
\end{description}

\section{Main Menu}
\label{sec:main_menu}

\section{Views}
\label{sec:application_views}
